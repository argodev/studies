\PassOptionsToPackage{unicode=true}{hyperref} % options for packages loaded elsewhere
\PassOptionsToPackage{hyphens}{url}
%
\documentclass[12pt,]{article}


\usepackage{lmodern}
\usepackage{setspace}
\setstretch{1.25}
\usepackage{amssymb,amsmath}
\usepackage{ifxetex,ifluatex}
\usepackage{fixltx2e} % provides \textsubscript
\ifnum 0\ifxetex 1\fi\ifluatex 1\fi=0 % if pdftex
  \usepackage[T1]{fontenc}
  \usepackage[utf8]{inputenc}
  \usepackage{textcomp} % provides euro and other symbols
\else % if luatex or xelatex
  \usepackage{unicode-math}
  \defaultfontfeatures{Ligatures=TeX,Scale=MatchLowercase}
\fi

% use upquote if available, for straight quotes in verbatim environments
\IfFileExists{upquote.sty}{\usepackage{upquote}}{}
% use microtype if available
\IfFileExists{microtype.sty}{%
\usepackage[]{microtype}
\UseMicrotypeSet[protrusion]{basicmath} % disable protrusion for tt fonts
}{}
\IfFileExists{parskip.sty}{%
\usepackage{parskip}
}{% else
\setlength{\parindent}{0pt}
\setlength{\parskip}{6pt plus 2pt minus 1pt}
}
\usepackage{hyperref}
\hypersetup{
            pdftitle={Class 12 - Ezekiel, Daniel},
            pdfauthor={Immanuel Church},
            pdfborder={0 0 0},
            breaklinks=true}
\urlstyle{same}  % don't use monospace font for urls
\usepackage[margin=1in]{geometry}


\setlength{\emergencystretch}{3em}  % prevent overfull lines
\providecommand{\tightlist}{%
  \setlength{\itemsep}{0pt}\setlength{\parskip}{0pt}}
\setcounter{secnumdepth}{0}
% Redefines (sub)paragraphs to behave more like sections
\ifx\paragraph\undefined\else
\let\oldparagraph\paragraph
\renewcommand{\paragraph}[1]{\oldparagraph{#1}\mbox{}}
\fi
\ifx\subparagraph\undefined\else
\let\oldsubparagraph\subparagraph
\renewcommand{\subparagraph}[1]{\oldsubparagraph{#1}\mbox{}}
\fi

% set default figure placement to htbp
\makeatletter
\def\fps@figure{htbp}
\makeatother

\usepackage{fancyhdr}
\pagestyle{fancy}
\fancyhead[RO,RE]{Class 12 - Ezekiel, Daniel}
\fancyhead[LO,LE]{Immanuel Church, Core Seminar}
\usepackage{tikz}
\usetikzlibrary{calc,shapes.multipart,chains,arrows}

\title{Class 12 - Ezekiel, Daniel}
\author{Immanuel Church}
\date{}

% REG: 180902
% custom definition of the title section
\usepackage{wrapfig}
\makeatletter
\def\@maketitle{%
  \newpage
  % \null
  \begin{wrapfigure}{l}{0.13\textwidth}
    \vspace{-\baselineskip}
    \includegraphics[scale=.35]{ImmanuelChurchSmall}
  \end{wrapfigure}

  \Large \@title \\
    
  \noindent\rule{\textwidth}{1pt}
  }
\makeatother


\begin{document}
      
    \maketitle
    
      
  \hypertarget{week-1-introduction-and-genesis-1-11}{%
\section{Week 1: Introduction and Genesis
1-11}\label{week-1-introduction-and-genesis-1-11}}

Welcome to the Old Testament core seminar! This is the beginning of 24
weeks through the whole Bible (12 this semester, 12 next). Ths class is
designed for you to profit even if you get just a few weeks here and
there. However, it's also designed such that the whole is greater than
the sum of its parts, because just as the whole Bible hangs together as
a single narrative, these classes all fit together. To that end, I'd
encourage you to try to come as often as possible and commit to taking
the time to study the entire Bible this Fall and Spring. I will warn you
that this course is derived from material that is 52 weeks in length,so
we will be moving quite quickly (52 weeks down to 24). Even so, it is an
opportunity to intentionally dive a bit deeper into God's word.



% some how, I'd like to force this to the bottom of the page
\bigbreak

\noindent\rule{\textwidth}{1pt}

\scriptsize{This material is adapted from a course with the same name developed by Capitol Hill Baptist Church. It has been modified for our purposes and has been condensed to fit our time schedule. The original version is available on their website at https://www.capitolhillbaptist.org/resources/core-seminars/series/old-testament-overview/}

%\let \footnote This material is now in the footnote

\end{document}
